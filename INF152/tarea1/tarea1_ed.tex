\documentclass[letterpaper,10pt]{article}
\usepackage[utf8]{inputenc}
\usepackage[spanish,mexico]{babel}
\usepackage{amsmath}
\usepackage{amsfonts}
\usepackage{amssymb}
\usepackage{enumerate}
\usepackage{float}
\usepackage{indentfirst}
\usepackage{graphicx}
\usepackage{url}
\usepackage{multicol}
\usepackage{geometry}
\usepackage{fullpage}
\usepackage{xcolor}
\usepackage{hyperref}

\usepackage{tikz}
%\usetikzlibrary{shapes}
\usetikzlibrary{arrows,automata}

\tikzset{
    %Define standard arrow tip
    >=stealth',
    % Define arrow style
    pil/.style={
           ->,
           thick,}
}

\def\checkmark{\tikz\fill[scale=0.4](0,.35) -- (.25,0) -- (1,.7) -- (.25,.15) -- cycle;}

\begin{document}

\thispagestyle{empty}
 	
\begin{minipage}[t]{0.6\textwidth}

{\LARGE \textbf{INF152} Estructuras Discretas - Tarea 1}

{\large \textbf{Profesores}: Claudio Lobos, Jorge Díaz Sebastián Gallardo, María Paz Vergara, Miguel Guevara.}\\
{\small \textbf{Ayudantes}: Valentina Aróstica, Bryan González, Sofía Mañana, Sofía Riquelme, Carla Herrera, Álvaro Gaete, Constanza Alvarado, Maciel Ripetti, Maureen Gavilán, Ignacio Jorquera, Fernanda Avendaño y Luis González.} 

Universidad T\'ecnica Federico Santa Mar\'{\i}a

Departamento de Inform\'atica -- SJ - CC 



\end{minipage}
\hfill



\vspace{0.3cm}


\begin{center}
    \huge Tarea 1
\end{center}


\section{Predicados}
\subsubsection{Formalizaci\'on de predicados}
\begin{minipage}[t]{0.4\textwidth}
\begin{itemize}
    \item \textit{A}(\textit{x}): \textit{x} le gusta leer.
    \item \textit{B}(\textit{x}): \textit{x} le gusta el café.
    \item \textit{C}(\textit{x}): \textit{x} le gusta la USM.
    \item \textit{D}(\textit{x}): \textit{x} es de Casa Central.
    \item \textit{E}(\textit{x}): \textit{x} juega basketball.
\end{itemize}
\end{minipage}
\vspace{0.4cm}
\newline
\begin{minipage}[t]{0.5\textwidth}
    Formalizando los enunciados obtenemos que:
    \begin{equation}
        ( \forall x \  A(x) \  \vee \  \forall x \  D(x) )\Rightarrow \forall x \  C(x)
    \end{equation}
    \begin{equation}
        \forall x \  E(x) \Leftrightarrow (\forall x \  \neg D(x) \wedge \forall x \  B(x) )
    \end{equation}
    \begin{equation}
        \exists x \  B(x) \Rightarrow E(x)
    \end{equation}
    \begin{equation}
       ( D(x) \vee \  \neg C(x) ) \Rightarrow B(x)
    \end{equation}
    \end{minipage}





\end{document}